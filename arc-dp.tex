%! program = pdflatex

\documentclass[12pt,a4paper]{article} 
\usepackage{arc-dp}
\usepackage{amsfonts}
\usepackage{amsmath}
\usepackage{graphicx}
\usepackage{times}
\usepackage{setspace}
\usepackage{fancyhdr}
\usepackage{color}
\usepackage[normalem]{ulem}
\usepackage{subfig}
\usepackage{float}
\usepackage{caption}
\usepackage{array}
\usepackage{pgfgantt}
\usepackage{url}
\usepackage{enumitem}
\usepackage{subfig}
\usepackage{pgfgantt}
\usepackage{wrapfig}
\usepackage{enumitem}

\usepackage{booktabs} % for spacing tables
\usepackage{tabularx} % auto table sizing
\usepackage{multirow} % table multirow

%\usepackage{epsf}
%\usepackage{fancyheadings}
%\usepackage{subfigure}
%\usepackage{pst-gantt}
%\usepackage{tweaklist}

\newcolumntype{L}[1]{>{\raggedright\let\newline\\\arraybackslash\hspace{0pt}}m{#1}}
\newcolumntype{C}[1]{>{\centering\let\newline\\\arraybackslash\hspace{0pt}}m{#1}}
\newcolumntype{R}[1]{>{\raggedleft\let\newline\\\arraybackslash\hspace{0pt}}m{#1}}

\let\OLDthebibliography\thebibliography
\renewcommand\thebibliography[1]{
  \OLDthebibliography{#1}
  \setlength{\parskip}{1pt}
  \setlength{\itemsep}{1pt plus 0.3ex}
}


%\renewcommand{\enumhook}{\setlength{\topsep}{0pt}%
 % \setlength{\itemsep}{-2mm}}
%\renewcommand{\itemhook}{\setlength{\topsep}{0pt}%
%  \setlength{\itemsep}{-2mm}}
  %%%%%UNCOMMENT THE NEXT COMMAND IF NEEDED
%\renewcommand{\descripthook}{\setlength{\topsep}{0pt}%
%  \setlength{\itemsep}{-2mm}}

%\pagestyle{fancy}

%\input{psfig.sty}
\newcommand{\todo}[1]{\textcolor{red}{#1}}
\newcommand{\rules}[1]{\textcolor{blue}{#1}}
\newcommand{\pset}{ {\rm P} \! \! \! {\rm P} }
\date{}
%\include{psfig}
\remove{
\topmargin -15mm
\headheight 0pt
\headsep 0pt
\textheight 285mm
\oddsidemargin -15mm
\evensidemargin -15mm
\textwidth 190mm
\columnsep 10mm
\marginparwidth 0pt
\marginparsep 0pt
}

\usepackage[top=0.5cm, bottom=0.5cm, left=0.5cm, right=0.5cm]{geometry}
\parindent=4mm
\parskip=0.2mm

%\usepackage{geometry} % see geometry.pdf on how to lay out the page. There's lots.
%\geometry{a4paper} % or letter or a5paper or ... etc
% \geometry{landscape} % rotated page geometry


%\linespread{1.5}

\newcommand*{\TitleFont}{%
      \usefont{\encodingdefault}{\rmdefault}{b}{n}%
      \fontsize{12}{12}%
      \selectfont}

\title{The Title of your ARC DP Proposal}
%\author{}
\date{} % delete this line to display the current date

%%% BEGIN DOCUMENT
\begin{document}
\rmfamily
\date{}


\noindent \textbf{PROJECT TITLE: }\\ \noindent The Title of your ARC DP Proposal

%Assisted Query Formulation for Cheaper, Faster & Unbiased Systematic Review


%Reducing Time and Cost for the Creation of Systematic Reviews through (Semi)-Automated Query Formulation\\ 
%\noindent (Semi-) Automatic Assisted Query Formulation for the Creation of Systematic Reviews

\subsection*{\TitleFont AIMS AND BACKGROUND}
\rules{ Briefly outline the aims and background of the proposal.  Include information about national/international progress in this field of research and its relationship to this proposal. Refer only to publications or non-traditional equivalents (outputs) that are accessible to the national and international research communities }
\subsubsection*{Example subsection: Background}

\subsection*{\TitleFont INVESTIGATORS}
\rules{This sub-section requires investigators to address the Selection Criteria (Investigators – 35\%). This section should be a high level summary of the Part D Personnel and ROPE section. Remember to:}

\rules{Ensure that all participants (CIs, PIs and other participants [such as RAs and technical staff]) in the proposal are described here, explaining how they will contribute to the project, including their roles, responsibilities and contributions.}

\rules{Provide evidence of research training, mentoring and supervision experience for each participant listed on the application.}

\rules{Describe investigator capacity to build international collaborations.}

\rules{Highlight that the project has the right team and design to achieve results with the time and resources available. This includes time and capacity to deliver, taking account of other grants held by the participants.}


\subsection*{\TitleFont PROJECT QUALITY AND INNOVATION}
\rules{This sub-section requires investigators to address the Selection Criteria (Project Quality and Innovation – 40\%). This section will need to provide: 
Detail around both significance and innovation}

\rules{An explanation of how the project will effectively address a significant problem}

\rules{Evidence that the framework is innovative and original}

\rules{Detail around the conceptual framework, design and methods to demonstrate that they are adequately developed, well integrated innovative and original.}

\rules{How the research will maximise benefits to Australia, and if relevant, how it addresses any Science and Research Priorities (and associated Practical Challenges).}

\rules{Describe the extent to which the proposal will advance knowledge.}

\rules{Describe the potential for the research to enhance international collaboration.}
\subsection*{Significance}
\rules{An explanation of how the project will effectively address a significant problem}
\subsection*{Innovation}
\rules{Evidence that the framework is innovative and original}
\subsection*{Conceptual Framework, Design and Methods}
\rules{Detail around the conceptual framework, design and methods to demonstrate that they are adequately developed, well integrated innovative and original.}
\subsubsection*{\underline{WP1: Example sub-sub-section, e.g. for work package}}

\subsection*{\TitleFont FEASIBILITY}

\rules{This sub-section requires investigators to address the Selection Criteria (Feasibility – 10\%). This section will need to provide: Specific detail around the feasibility of the project in terms of resources, availability of facilities and intellectual capacity, in order to ensure the project can be completed within budget and timeframe. Demonstration of a supportive and high quality environment for this project and for HDR students (where appropriate). A timeline of project activities may be useful in this section.}
\subsubsection*{\underline{Timeframe and Budget}}
 \subsubsection*{\underline{Supportive and high quality research environment}} 
  \subsubsection*{\underline{Expertise and Intellectual Capacity}} 

\subsection*{\TitleFont BENEFIT}
\rules{This sub-section requires investigators to address the Selection Criteria – Benefit (15\%). This includes : Outline how the completed project will produce significant new knowledge and/or innovative economic, commercial, environmental, social and/or cultural benefit to the Australian and international community. 
This response should expand on the National Interest Test Statement provided at A6. 
Demonstrate how the project will be cost-effective and value for money.
}


\subsection*{\TitleFont COMMUNICATION OF RESULTS}
\rules{Outline plans for communicating the research results to other researchers and the broader community, including scholarly and public communication and dissemination. This could also include plans for commercialisation.}

\subsection*{\TitleFont MANAGEMENT OF DATA}
\rules{ Outline plans for the management of data produced as a result of the proposed research, including but not limited to storage, access and re-use arrangement.
 It is not sufficient to state that the organisation has a data management policy. Researchers are encouraged to highlight specific plans for the management of their research data. Please see UQ Data Management Tip Sheet \url{https://research.uq.edu.au/research-support/research-management/funding-schemes/australian-research-council-arc/arc-discovery-projects} for advice on resources available to support data management.
}

%\vspace{4pt}
\renewcommand{\refname}{\normalfont\selectfont\TitleFont REFERENCES} 
%\vspace{-14pt}
\begingroup
    \fontsize{10pt}{10pt}\selectfont
\bibliographystyle{abbrv}
\bibliography{references.bib}

%\todo{\subsection*{Changes to be made}}
%\todo{
%\begin{enumerate}
%\item bring clear aims, objectives and methods in the first page
%\item clarify how the method is unique, why cannot be done by Google or others. Clarify why it leads to a solution
%\item reduce the quantity of text
%\item add images: 1. example of good query and results vs example of bad query and result; 2. study methodology/phases
%\end{enumerate}}


\endgroup


\end{document}


